% ========================================
% Example Slides - Customize as needed
% ========================================

% ===== Table of Contents =====
\begin{frame}{Table of Contents}
  \tableofcontents
\end{frame}

% ========================================
% Section 1: Basic Features
% ========================================
\section{Basic Features}

\begin{frame}{Basic Text and Lists}
  \begin{itemize}
    \item First main point
    \begin{itemize}
      \item Sub-point one
      \item Sub-point two
    \end{itemize}
    \item Second main point
    \item Third main point
  \end{itemize}
  
  \vspace{1em}
  
  Numbered lists:
  \begin{enumerate}
    \item First numbered item
    \item Second numbered item
    \item Third numbered item
  \end{enumerate}
\end{frame}

\begin{frame}{Text Highlighting and Alerts}
  In this slide, some important text will be \alert{highlighted in orange} because it's important.
  
  \vspace{1em}
  
  You can also use \textbf{bold text}, \textit{italic text}, or \underline{underlined text} for emphasis.
  
  \vspace{1em}
  
  \begin{block}{Information Block}
    This is a standard information block. Use it to highlight key concepts or definitions.
  \end{block}
\end{frame}

% ========================================
% Section 2: Tables and Figures
% ========================================
\section{Tables and Figures}

\begin{frame}{Professional Tables}
  \begin{table}
    \centering
    \caption{Example data table}
    \begin{tabular}{lcc}
      \toprule
      \textbf{Method} & \textbf{Accuracy (\%)} & \textbf{Speed (ms)} \\
      \midrule
      Method A & 85.3 & 12.5 \\
      Method B & 87.1 & 15.2 \\
      Method C & 89.4 & 18.7 \\
      \bottomrule
    \end{tabular}
  \end{table}
\end{frame}

\begin{frame}{Including Images}
  \begin{figure}
    \centering
    \includegraphics[width=0.6\linewidth]{sample-photo.pdf}
    \caption{Example figure with caption}
  \end{figure}
\end{frame}

\begin{frame}{Side-by-Side Figures}
  \begin{figure}
    \centering
    \begin{minipage}{0.48\textwidth}
      \centering
      \includegraphics[width=0.9\linewidth]{sample-diagram.pdf}
      \caption{First image}
    \end{minipage}
    \hfill
    \begin{minipage}{0.48\textwidth}
      \centering
      \includegraphics[width=0.9\linewidth]{sample-plot.pdf}
      \caption{Second image}
    \end{minipage}
  \end{figure}
\end{frame}

% ========================================
% Section 3: Advanced Features
% ========================================
\section{Advanced Features}

\begin{frame}{Two-Column Layout}
  \begin{columns}
    \begin{column}{0.5\textwidth}
      \textbf{Left Column}
      \begin{itemize}
        \item Point one
        \item Point two
        \item Point three
      \end{itemize}
      
      $$E = mc^2$$
    \end{column}
    
    \begin{column}{0.5\textwidth}
      \textbf{Right Column}
      \begin{itemize}
        \item Different point
        \item Another point
        \item Final point
      \end{itemize}
      
      $$\sum_{i=1}^{n} i = \frac{n(n+1)}{2}$$
    \end{column}
  \end{columns}
\end{frame}

\begin{frame}{Colored Boxes with TColorBox}
  \begin{tcolorbox}[colback=lightorange, colframe=orange, title=Important Note]
    This is a colored text box that can be used to highlight important information, 
    intuitions, or key takeaways from your presentation.
  \end{tcolorbox}
  
  \vspace{0.5em}
  
  \begin{tcolorbox}[colback=blue!5!white, colframe=blue!75!black, title=Technical Detail]
    You can customize the colors and styles of these boxes to match your presentation theme.
  \end{tcolorbox}
\end{frame}

% ========================================
% Section 4: Citations and References
% ========================================
\section{Citations and References}

% Example 1: Using refsection for per-slide references
\begin{refsection}
\begin{frame}{Slide with References (Method 1)}
  This slide demonstrates how to cite papers and display references at the bottom.
  
  \vspace{1em}
  
  Important research by~\parencite{sample2024} shows that this method is effective. 
  Another study~\parencite{another2023} confirms these findings.
  
  \vspace{1em}
  
  The references will appear at the bottom left of this slide.
  
  % Display references at bottom-left
  \bottomleftrefs
\end{frame}
\end{refsection}

% Example 2: Manual references using setSlideBib
\begin{frame}{Slide with Manual Reference (Method 2)}
  % Manually set the reference text for this slide
  \setSlideBib{Smith et al., Important Paper Title, Conference 2024}
  
  This method allows you to manually specify reference text without using biblatex citations.
  
  \vspace{1em}
  
  Useful when you want more control over the reference formatting or when the citation 
  isn't in your .bib file.
  
  % Note: The reference will appear at the bottom left of this slide
\end{frame}

% Clear the manual reference for subsequent slides
\clearSlideBib

% ========================================
% Section 5: Multi-page PDF Figures
% ========================================
\section{Multi-page PDF Figures}

\begin{frame}{Using Specific Pages from Multi-page PDFs}
  You can include specific pages from multi-page PDF files using the \texttt{page} option:
  
  \begin{figure}
    \centering
    % Example: Including page 1 from a sample multi-page PDF
    \includegraphics[width=0.7\linewidth, page=1]{sample-multipage.pdf}
    \caption{Page 1 of multi-page PDF (sample-multipage.pdf)}
  \end{figure}
  
  \vspace{0.5em}
  \small
  Change the \texttt{page=} number to display different pages from the PDF.
\end{frame}

\begin{frame}{Cropping and Adjusting PDF Pages}
  Use \texttt{adjustbox} to crop or trim specific regions:
  
  \begin{figure}
    \centering
    % Example: crop top and bottom margins, show only middle content
    \adjustbox{width=0.8\linewidth, trim=0 {0.125\height} 0 {0.125\height}, clip}{
      \includegraphics[page=2]{sample-multipage.pdf}
    }
    \caption{Page 2 of sample-multipage.pdf, with margins cropped}
  \end{figure}
  
  \vspace{0.5em}
  \small
  Trim syntax: \texttt{trim=left bottom right top, clip}\\
  Use absolute values (e.g., \texttt{10pt}) or relative (e.g., \texttt{\{0.2\textbackslash height\}})
\end{frame}

% ========================================
% Section 6: Mathematical Equations
% ========================================
\section{Mathematical Equations}

\begin{frame}{Inline and Display Math}
  Inline math: $f(x) = ax^2 + bx + c$
  
  \vspace{1em}
  
  Display math (centered):
  $$\int_{-\infty}^{\infty} e^{-x^2} dx = \sqrt{\pi}$$
  
  \vspace{1em}
  
  Aligned equations:
  \begin{align*}
    \nabla \times \mathbf{E} &= -\frac{\partial \mathbf{B}}{\partial t} \\
    \nabla \times \mathbf{H} &= \mathbf{J} + \frac{\partial \mathbf{D}}{\partial t}
  \end{align*}
\end{frame}

% ========================================
% Final Slide: References Page
% ========================================
\section*{References}

% Cite all bibliography entries to ensure they appear in the References slide
\nocite{*}

\begin{frame}{References}
  \scriptsize
  % Print all references
  \printbibliography[heading=none]
\end{frame}

% ========================================
% Thank You Slide
% ========================================
\begin{frame}[plain]
  \centering
  \vspace{2cm}
  
  \textbf{\Huge Thank You}
  
  \vspace{1cm}
  
  \large Questions?
  
  \vspace{1cm}
  
  \normalsize
  \textbf{Contact}\\
  \texttt{your.email@example.com}
  
\end{frame}

